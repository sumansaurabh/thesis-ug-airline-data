The main thrust of this paper is on visual analysis of air traffic data. Hence, this section focuses on work related to visualizing air traffic data. One of the most popular technique for visualizing air traffic data is to represent the trajectory of each aircraft as an animated particle. Many such visualizations are available on the web via sites such as youtube. A version that was designed by Aaron Koblin demonstrates several techniques and embellishments for presenting the flight trajectories. More recently, the discrete nature of the flight tracks were smoothed out to obtain a continuous estimate of air traffic density using a view dependent kernel density estimator. Representing air traffic data as a density plot is not new. Kellner [8] also used density plots of the arrival and departure rates of aircraft at different airports to assess their capacity. This paper will use similar techniques in visualizing the air traffic data. More importantly, our work examines the impact of varying minimum aircraft separation policy on air traffic density, and also examines if a flight plan, e.g. of a UAV operation request, will endanger existing flight patterns.

There are many factors affecting air traffic congestion and airport capacity. One of those that is controllable and fall under policy decisions is the specification of minimum separation between aircraft. Currently, this is set to 5 nautical miles horizontally, and 1,000
feet vertically [4] when the aircraft is en-route. This limit is adjusted as the aircraft approaches an airport and can drop to 3 miles horizontally on landing approaches to airports. The relative weight class of the leading and following aircraft are also taken into con-
sideration in such situations in order to reduce risks due to wake turbulence [3]. The en-route limit accounts for aircraft speed (typical passenger jets fly at average speed of 500 miles per hour or just over 8 miles per minutes), weather impact on visibility, and wake turbulence from leading aircraft, among other factors. With the touted capabilities of ADS-B, the NextGen enabled weather system, and integrated information system, one can theoretically safely reduce the minimum separation requirements between aircraft. This paper provides visual analysis tools to examine the effects of different shapes and parameters describing the minimum separation volume between aircraft.

With regards to UAV operation, they are more generally referred to as Unmanned Aircraft Systems (UAS)[7, 2]. Over the past few years, interest in UAS has rapidly increased. This is because of the possibilities they offer to both government and commercial interests. They would enable a broad range of satellite-like abilities, but at a much lower cost. Aerial photography, communications, environmental monitoring, and security are some of the abilities that UAS deployment could make possible on a large scale. Currently, UAS are predominantly used by the Department of Defense and the Department of Homeland Security, and often outside of national air space (NAS). A handful of UAS are allowed to operate inside our NAS, though almost exclusively for national security or research purposes. However, each UAS operation must be pre-approved by the FAA on a case by case basis. This process is very tedious and does not scale well to large numbers of flights. There are a few studies on risk managment of operating UAS. A recent study uses a site-specific non- uniform probabilistic background air traffic to study the risks [11]. Using the visual analysis tools presented in this paper, checking whether the flight plan for a UAS will allow for a safe operation within the NAS can be accomplished expeditiously.