Changes in the patterns of air traffic at New York and Chicago over 21 years (from 1987 to 2008) are explored. Most notably rise and fall of curve at 9/11 in delay and total traffic respectively. Befor 1996, the Chicago was
completely gone, but a slow recovery started to take place. The levels of air traffic were more or less stable
over the five year period ending in 2008. The population growth of the surrounding area only seems to correlate
with air traffic levels in the post-hub time period. Also, as more airlines served the airport, more destinations were added. However, delays for both arrivals and departures increased as well.


\section{Scope of further work}
There are many possible future investigations that could be pursued. One would be to compare trends with national trends. The analysis performed here could even be replicated for other airports around the country.
It would also be interesting to get passenger data from the airline traffic that could help shed more light on the validity of the speculative explanations for the trends presented their. With regards to the delay
data, adding weather data could be very interesting in helping explain some of the variation in delays. There are also more data than delay and cancel rate in the flight data provided by the Data Expo 2009 competition that could be further explored, especially in the on-time performance area. What airline, time of day, day of the week, month of the year should you choose to travel on to minimize the chance of being delayed at USA Air Traffic?