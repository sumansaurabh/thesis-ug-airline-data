\section{The Area of Work}
 
\blindtext


\section{Problem Addressed}
Aviation and air travel has established itself as a key economic and social resource in modern times. As the world population increases and becomes ever more interconnected, the demand for air travel will only increase. Currently there are over 100,000 commercial aviation flights and over 200,000 general aviation flights within the national airspace system (NAS) every day. This does not include military sorties or other, special purpose, flights within the NAS. The number of passengers flying to or from the U.S. is expected to grow an
average of 4.5\% annually, with cargo amounts showing a similar increase, while general aviation is expected to
grow 1\% annually. In addition, there is increasing interest, from both government and commercial sectors, in integrating unmanned aerial vehicles (UAV’s) into the NAS. Though full UAV integration poses its own unique set of complications, nevertheless it is only a matter of time before they contribute to the air traffic over the NAS. This constant increase in air traffic within the increasingly congested NAS will require new methods and techniques to efficiently accommodate new traffic.

To address these issues, the US Congress approved plans for the development of the Next Generation Air Transportation System (NextGen). It is an overhaul of the current NAS with the goals of allowing more aircraft to safely fly closer together with more direct
routes. It is scheduled for implementation in stages between 2012 and 2025 with 5 major elements: (i) Automatic dependent surveillance-broadcast (ADS-B) will replace radar systems with satellite based global positioning information for each aircraft. This infor- mation will be broadcast in realtime to airports another aircraft within a 150 mile radius allowing them to fly closer without jeopardizing safety. (ii) Systemwide information management (SWIM) is a consolidation of multiple information systems into a single coherent system and will reduce redundancy and facilitate information sharing. (iii) NextGen data communication will add data links between aircraft and air traffic controllers to the current two-way voice communication. (iv) NextGen network enabled weather is an ambitious effort to fuse data from tens of thousands of ground, air, and space based sensors into a single national weather information system to provide realtime weather information. (v) NAS voice switch (NVS) will replace multiple existing voice switching systems into a single consolidated air/ground and ground/ground voice communication system.The NextGen system will provide the infrastructure to allow aircraft to safely fly closer together thereby making more efficient use of limited airspace. It will allow aircraft to use more direct routes instead of being constrained to predetermined sky highways thereby
reducing congestion and reduce fuel costs. With pieces of the NextGen infrastructure coming into place, there is an opportunity to further their benefits by developing software tools that provide added value.

 This paper focuses on visual analysis tools to study the changes on air traffic congestion in span of 21 years which would allow policy makers to see the effects of changing the aircraft separation volume on congestion. The same tool can also be used as a decision aid for processing requests for unmanned aerial vehicle operations. Specifically, this paper will discuss methods
and tools used to calculate and render air traffic densities over areas of interest, as well as methods for aggregating such traffic densities over different time scales to extract fluctuations and periodic cycles in traffic patterns. We apply these tools to study the effects of possible modifications to the current en-route aircraft separation requirements. These modification, which are based on the characteristics of large fixed wing aircraft, has the potential of increasing the amount of available air space, allowing for future increases in overall air traffic numbers. In addition, we apply the same suite of tools to provide a quick visual inspection of planned UAV operation under different aircraft separation requirements. The studies conducted in this paper are based on a data set which is constructed from information made available by the Bureau of Transportation Statistics, 

There are over 300,000 flightswithin the United States every day. In the future, daily air traffic number of all varieties are expected to continue rising. In addition, there is increasing interest in integrating unmanned aerial vehicles, for both government and commercial interests, into the national airspace system (NAS). This large growth in aviation operations will only increase traffic within the already limited NAS, leading to higher congestion and less free airspace. In this report, we present visual analysis tools
to study the changes on air traffic congestion in span of 21 years. The tools support visualization of time-varying
air traffic density over an area of interest using different time granularity. We use this visual analysis
platform to investigate how changing the aircraft separation volume can reduce congestion while maintaining key safety requirements. The same tool can also be used as a decision aid for processing requests for unmanned aerial vehicle operations.

To present our analysis on traffic we utilized Airline Data presented at ASA Data Expo 2009. This dataset is constructed from information made available by the Bureau of Transportation Statistics, USA. It consists of more than 120 million records corresponding to each commercial airline flight in the United States between 1987 and 2008. As datasets gets larger, real-time visualization becomes more difficult. Supposedly a dataset with a billion entries. If we compute a summary of the dataset and visualize it we will either need non-trivial parallel rendering algorithms or significant time to produce a drawing. This solutions would not scale well. To perform analysis we need to mine relevant data using MapReduce programming. 
 

\subsection{Data}
The data comes originally from RITA where it is described in detail. It can download the data there, or from the bzipped csv files listed below. These files have derivable variables removed, are packaged in yearly chunks and have been more heavily compressed than the originals.

\begin{tabular}{llll}
	Variable & Description & Variable & Description\\
	Year & 1987-2008 & DepDelay & departure delay, in minutes\\
	Month & 1-12 & Origin & 	origin IATA airport code\\
	DayofMonth & 1-31 & Dest & 	destination IATA airport code\\
	DayOfWeek & 1 (Monday) - 7 (Sunday) & Distance & in miles\\
	DepTime & actual departure time (local, hhmm) & TaxiIn & taxi in time, in minutes\\
	CRSDepTime & scheduled departure time (local, hhmm) & TaxiOut & taxi out time in minutes\\
	ArrTime & actual arrival time (local, hhmm) & Cancelled & was the flight cancelled?\\
	CRSArrTime  & scheduled arrival time (local, hhmm) & CancellationCode & 	reason for cancellation (A = carrier, B = weather, C = NAS, D = security)\\
	UniqueCarrier & 	unique carrier code & Diverted & 1 = yes, 0 = no\\
	FlightNum & flight number & CarrierDelay & in minutes\\
	TailNum & plane tail number & WeatherDelay & in minutes\\
	ActualElapsedTime  & in minutes & NASDelay & in minutes\\
	CRSElapsedTime &  	in minutes & SecurityDelay & in minutes\\
	AirTime & in minutes & LateAircraftDelay & in minutes\\
	ArrDelay & arrival delay, in minutes &  & \\
\end{tabular}






\subsection{Goal}
This is intentionally vague in order to allow different entries to focus on different aspects of the data, but here are a few ideas to that we focussed on :
\vspace*{-2mm}
\begin{description}
  \item[-] Summarize data by time periods, airport, and carrier

  \vspace*{-4mm}

  \item[-] Temporal effects
  \vspace*{-4mm}
  \begin{description}
  \item[-] Are some time periods more prone to delays than others?
  \vspace*{-3mm}
  \item[-] Relationships between delays and \textit{Seasonal factors}: winter, summer, holidays \textit{Weather factors}: Blizzards and severe weather \textit{Daily factors}: Time of day, day of week
  \end{description}
  \vspace*{-4mm}
  \item[-] Spatial effects
  \vspace*{-2mm}
  \begin{description}
  \item[-] Are some airports more prone to delays than others?
  \end{description}
  \vspace*{-2mm}
  \item[-] Carrier effects
  \begin{description}
    \vspace*{-2mm}
  \item[-] Are some carriers more prone to delays than others?
  \end{description}

  \vspace*{-4mm}
  \item[-] Analysis of traffic on New York and Chicago a densly populated metropolitan cities in USA?

\end{description}

\subsection{Tools Used} % (fold)
Rhipe packages are used for the development of the MapReduce modal. R is used for Visualization along with googleVis and Shiny app to make it interactive.

\subsubsection{R}
R is a programming language and software environment for statistical computing and graphics. The R language is widely used among statisticians and data miners for developing statistical software and data analysis.

\subsubsection{Rhipe}
RHIPE is an R package that provides a way to use Hadoop from R. It can be used on its own or as part of the Tessera environment. RHIPE (hree-pay') is the R and Hadoop Integrated Programming Environment. RHIPE is a merger of R and Hadoop. R is the widely used, highly acclaimed interactive language and environment for data analysis. Hadoop consists of the Hadoop Distributed File System (HDFS) and the MapReduce distributed compute engine. RHIPE allows an analyst to carry out D\&R analysis of complex big data wholly from within R. RHIPE communicates with Hadoop to carry out the big, parallel computations. 

\subsubsection{ggplot}
ggplot2 is a plotting system for R, based on the grammar of graphics, which tries to take the good parts of base and lattice graphics and none of the bad parts. It takes care of many of the fiddly details that make plotting a hassle (like drawing legends) as well as providing a powerful model of graphics that makes it easy to produce complex multi-layered graphics.







\section{Existing System}
The main thrust of this paper is on visual analysis of air traffic data. Hence, this section focuses on work related to visualizing air traffic data. One of the most popular technique for visualizing air traffic data is to represent the trajectory of each aircraft as an animated particle. Many such visualizations are available on the web via sites such as youtube. A version that was designed by Aaron Koblin demonstrates several techniques and embellishments for presenting the flight trajectories. More recently, the discrete nature of the flight tracks were smoothed out to obtain a continuous estimate of air traffic density using a view dependent kernel density estimator. Representing air traffic data as a density plot is not new. Kellner [8] also used density plots of the arrival and departure rates of aircraft at different airports to assess their capacity. This paper will use similar techniques in visualizing the air traffic data. More importantly, our work examines the impact of varying minimum aircraft separation policy on air traffic density, and also examines if a flight plan, e.g. of a UAV operation request, will endanger existing flight patterns.

There are many factors affecting air traffic congestion and airport capacity. One of those that is controllable and fall under policy decisions is the specification of minimum separation between aircraft. Currently, this is set to 5 nautical miles horizontally, and 1,000
feet vertically [4] when the aircraft is en-route. This limit is adjusted as the aircraft approaches an airport and can drop to 3 miles horizontally on landing approaches to airports. The relative weight class of the leading and following aircraft are also taken into con-
sideration in such situations in order to reduce risks due to wake turbulence [3]. The en-route limit accounts for aircraft speed (typical passenger jets fly at average speed of 500 miles per hour or just over 8 miles per minutes), weather impact on visibility, and wake turbulence from leading aircraft, among other factors. With the touted capabilities of ADS-B, the NextGen enabled weather system, and integrated information system, one can theoretically safely reduce the minimum separation requirements between aircraft. This paper provides visual analysis tools to examine the effects of different shapes and parameters describing the minimum separation volume between aircraft.

With regards to UAV operation, they are more generally referred to as Unmanned Aircraft Systems (UAS)[7, 2]. Over the past few years, interest in UAS has rapidly increased. This is because of the possibilities they offer to both government and commercial interests. They would enable a broad range of satellite-like abilities, but at a much lower cost. Aerial photography, communications, environmental monitoring, and security are some of the abilities that UAS deployment could make possible on a large scale. Currently, UAS are predominantly used by the Department of Defense and the Department of Homeland Security, and often outside of national air space (NAS). A handful of UAS are allowed to operate inside our NAS, though almost exclusively for national security or research purposes. However, each UAS operation must be pre-approved by the FAA on a case by case basis. This process is very tedious and does not scale well to large numbers of flights. There are a few studies on risk managment of operating UAS. A recent study uses a site-specific non- uniform probabilistic background air traffic to study the risks [11]. Using the visual analysis tools presented in this paper, checking whether the flight plan for a UAS will allow for a safe operation within the NAS can be accomplished expeditiously.


\subsection{The  Oracle Airline Data Model}
The  Oracle  Airline Data Model is a powerful logical and physical data model that will help  airlines effectively store, manage, and analyze airline data that currently resides in passenger service systems (includes
reservation systems and departure control systems), global distribution system (GDS), loyalty management systems, and customer data warehouses.  It provides a single scalable repository for transactional and historical data 
that can be used to provide real-time business intelligence and strategic 
insights you’re  your airline. Using sophisticated trending and data mining capabilities based on Oracle’s  OLAP and data mining technology,  airline personnel will now have the data analysis  capabilities to develop Airline -specific insights that are relevant, actionable, and can improve both top-line and bottom-line results. 

The Oracle Airline Data Model provides detail transaction storage and advanced analysis into a full range of airline
subject areas, including reservations, sales, operations, loyalty, 
and finance.  Using reservation data, the data model can provide detailed insight into passenger bookings by time period, fare class, and flight.  It provides insights into channel performance looking at bookings, cancellations, and revenues through travel agency, OTA, ticket counter, call center, and web channels.  It allows you to analyze passenger 
revenues by geography, time period, and flight.  Finally it provides insights into loyalty program member activity through a variety of reports.  
The data model fits the needs of large network carriers and low-cost carriers. 

 
\section{Creation of bibliography}
Use sampleBib.bib file to save your bib format citations. Use the command \cite{saini2010alternative} for referring to a particular article. 





