Have you ever been stuck in a traffic and wondered you could have predicted the traffic pattern if you'd had more data? Respectfully digital India is evolving and getting enough traffic data of either Road, Railway or Airline would take years.

So to present our analysis on traffic we utilized Airline Data presented at ASA Data Expo 2009. This dataset is constructed from information made available by the Bureau of Transportation Statistics, USA. It consists of more than 120 million records corresponding to each commercial airline flight in the United States between 1987 and 2008\cite{dataexpo2009}.

As datasets get larger, real-time visualization becomes more difficult. Supposedly a dataset with a billion entries. If we compute a summary of the dataset and visualize it we will either need non-trivial parallel rendering algorithms or significant time to produce a drawing. This solutions would not scale well. To perform analysis we need to mine relevant data using MapReduce programming. 

Airline dataset, because of it's large size we used Hadoop MapReduce to mine the relevant data. Graphical summary were than built in R and analysis were performed on it. It revealed the changing patterns in the flight traffic, cancellation, delays with a spot light on days after 9/11. 

Before delving into analytics of airline data, we had a literary overview of how Data Analytics is transforming the digital world using the tools like MapReduce, Amazon EMC. The ongoing importance of consuming and extrapolating “Total Data” for new business processes and analytics approach has brought us tools like Hadoop, Spark that provides pragmatic, cost-effective, scalable infrastructure for building analytic solution on Big Data.

Big Data has not only changed the tools one can use for predictive analytics, it also changed the entire way of thinking about knowledge extraction and interpretation. Traditionally, data science has always been dominated by trial-and-error analysis, an approach that becomes impossible when datasets are large and heterogeneous.
     